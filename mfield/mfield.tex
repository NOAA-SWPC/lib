\documentclass{article}

\usepackage{amsmath}
\usepackage{amstext}
\usepackage{array}

\begin{document}

\section{Introduction}

\section{Internal field}
\label{sec:internal}

The internal scalar potential is given in geocentric spherical coordinates by
\begin{equation}
V_{int}(r,\theta,\phi) = a \sum_{nm} \left( \frac{a}{r} \right)^{n+1} \left( g_{nm} \cos{m \phi} + h_{nm} \sin{m \phi} \right)
P_{nm}(\cos{\theta})
\end{equation}
Defining $\mathbf{B} = - \nabla V$ gives
\begin{align}
B_x = -B_{\theta} = \frac{1}{r} \frac{\partial V}{\partial \theta} &= \sum_{nm} \left( \frac{a}{r} \right)^{n+2}
\left( g_{nm} \cos{m \phi} + h_{nm}\sin{m \phi} \right) \frac{\partial}{\partial \theta} P_{nm}(\cos{\theta}) \\
B_y = B_{\phi} = -\frac{1}{r \sin{\theta}} \frac{\partial V}{\partial \phi} &= \frac{1}{\sin{\theta}} \sum_{nm}
\left( \frac{a}{r} \right)^{n+2} m \left( g_{nm} \sin{m \phi} - h_{nm} \cos{m \phi} \right) P_{nm}(\cos{\theta}) \\
B_z = -B_r = \frac{\partial V}{\partial r} &= -\sum_{nm} (n + 1) \left( \frac{a}{r} \right)^{n+2}
\left( g_{nm} \cos{m \phi} + h_{nm}\sin{m \phi} \right) P_{nm}(\cos{\theta})
\end{align}

\section{Complex Internal field}
\label{sec:internal_complex}

The complex internal scalar potential is given in geocentric spherical coordinates by
\begin{equation}
V_{int}(r,\theta,\phi) = a \sum_{nm} \left( \frac{a}{r} \right)^{n+1} g_{nm} Y_{nm}(\theta,\phi)
\end{equation}
where the coefficients $g_{nm}$ are complex and $Y_{nm} = P_{nm}(\cos{\theta}) \exp{im\phi}$.
Defining $\mathbf{B} = - \nabla V$ gives
\begin{align}
B_x = -B_{\theta} = \frac{1}{r} \frac{\partial V}{\partial \theta} &= \sum_{nm} \left( \frac{a}{r} \right)^{n+2}
g_{nm} \frac{\partial}{\partial \theta} Y_{nm} \\
B_y = B_{\phi} = -\frac{1}{r \sin{\theta}} \frac{\partial V}{\partial \phi} &= -\frac{1}{\sin{\theta}} \sum_{nm}
im \left( \frac{a}{r} \right)^{n+2} g_{nm} Y_{nm} \\
B_z = -B_r = \frac{\partial V}{\partial r} &= -\sum_{nm} (n + 1) \left( \frac{a}{r} \right)^{n+2} g_{nm} Y_{nm}
\end{align}

\section{External field}

The external scalar potential is given in geocentric spherical coordinates by
\begin{equation}
V_{ext}(r,\theta,\phi) = a \sum_{nm} \left( \frac{r}{a} \right)^n \left( q_{nm} \cos{m \phi} + k_{nm} \sin{m \phi} \right)
P_{nm}(\cos{\theta})
\end{equation}

The magnetic field components are
\begin{align}
B_x &= \sum_{nm} \left( \frac{r}{a} \right)^{n-1} \left( q_{nm} \cos{m \phi} + k_{nm} \sin{m \phi} \right)
\frac{\partial}{\partial \theta} P_{nm}(\cos{\theta}) \\
B_y &= \frac{1}{\sin{\theta}} \sum_{nm} \left( \frac{r}{a} \right)^{n-1} m
\left( q_{nm} \sin{m \phi} - k_{nm} \cos{m \phi} \right) P_{nm}(\cos{\theta}) \\
B_z &= \sum_{nm} n \left( \frac{r}{a} \right)^{n-1} \left( q_{nm} \cos{m \phi} + k_{nm} \sin{m \phi} \right)
P_{nm}(\cos{\theta})
\end{align}

\section{Complex External field}

The complex external scalar potential is given in geocentric spherical coordinates by
\begin{equation}
V_{ext}(r,\theta,\phi) = a \sum_{nm} k_{nm} \left( \frac{r}{a} \right)^n Y_{nm}(\theta,\phi)
\end{equation}

The magnetic field components are
\begin{align}
B_x &= \sum_{nm} k_{nm} \left( \frac{r}{a} \right)^{n-1} \frac{\partial}{\partial \theta} Y_{nm} \\
B_y &= -\frac{1}{\sin{\theta}} \sum_{nm} im k_{nm} \left( \frac{r}{a} \right)^{n-1} Y_{nm} \\
B_z &= \sum_{nm} n k_{nm} \left( \frac{r}{a} \right)^{n-1} Y_{nm}
\end{align}

\section{Modeling}

The penalty function which is minimized is
\begin{equation}
\chi^2 = \sum_{i=1}^{N_{vec}} \boldsymbol{\epsilon}_i \cdot \boldsymbol{\epsilon}_i + \sum_{i=1}^{N_{scal}} f_i^2
\end{equation}
where the residuals are
\begin{align}
\boldsymbol{\epsilon}_i & = R_q R_3(\alpha,\beta,\gamma) \mathbf{B}^{VFM}_i - \mathbf{B}^{model}(\mathbf{g},\mathbf{k}) \\
f_i & = || \mathbf{B}^{model}(\mathbf{g},\mathbf{k}) || - F_i
\end{align}
Here, $\mathbf{B}^{VFM}_i$ is the vector measurement in the VFM instrument frame and $F_i$ is the scalar field
measurement. The matrix $R_3(\alpha,\beta,\gamma)$ rotates a vector from the VFM frame to the CRF frame defined
by the star camera using the Euler angles $\alpha,\beta,\gamma$. The matrix $R_q$ then rotates from CRF to NEC
(see Olsen et al, 2013). The vector model is given by
\begin{equation}
\mathbf{B}^{model}(\mathbf{g},\mathbf{r}) = \mathbf{B}^{int}(\mathbf{g}) + \mathbf{B}^{crust} + \mathbf{B}^{ext,pomme} + \mathbf{B}^{ext,correction}(\mathbf{k})
\end{equation}
where $\mathbf{B}^{int}(\mathbf{g})$ is an internal field model defined in Sec.~\ref{sec:internal},
$\mathbf{B}^{crust}$ is the MF7 crustal field model from degree 16 to 133,
$\mathbf{B}^{ext,pomme}$ is the POMME-8 external field model, and $\mathbf{B}^{ext,correction}(\mathbf{k})$ is a daily
ring current correction, parameterized as
\begin{equation}
\mathbf{B}^{ext,correction}(\mathbf{k}) = k(t) \left( 0.7 \mathbf{B}^{ext,dipole} + 0.3 \mathbf{B}^{int,dipole} \right)
\end{equation}
where $\mathbf{B}^{ext,dipole}$ and $\mathbf{B}^{int,dipole}$ are degree 1 external and internal dipole fields,
whose coefficients are aligned with the main field dipole (ie: $g_{10},g_{11},h_{11}$). For each day, there is
one coefficient $k(t)$ for that day which is determined through the least
squares minimization. The crustal field term, $\mathbf{B}^{crust}$, can
optionally be set to zero, in order to fit a high degree crustal field
in $\mathbf{B}^{int}$.

\subsection{Jacobian}

When minimizing $\chi^2$ with a nonlinear least squares algorithm, the Jacobian
is required.

\subsubsection{Required derivatives}

For easy reference, we list the derivatives of the residuals with respect
to various model parameters, needed for the Jacobian calculation. Here, we
assume that the internal field model can be expressed as
\begin{equation}
\mathbf{B}^{int}(\mathbf{r}; \mathbf{g}) = \sum_{nm} g_{nm} d\mathbf{B}^{int}_{nm}(\mathbf{r})
\end{equation}
where
\begin{equation}
d\mathbf{B}^{int}_{nm}(\mathbf{r}) =
\left\{
\begin{array}{cc}
\left( \frac{a}{r} \right)^{n+2}
\left(
\begin{array}{c}
\cos{(m\phi)} \partial_{\theta} P_{nm} \\
\frac{m}{\sin{\theta}} \sin{(m\phi)} P_{nm} \\
-(n+1) \cos{(m\phi)} P_{nm} \\
\end{array}
\right) & m \ge 0 \\
\left( \frac{a}{r} \right)^{n+2}
\left(
\begin{array}{c}
\sin{(m\phi)} \partial_{\theta} P_{nm} \\
-\frac{m}{\sin{\theta}} \cos{(m\phi)} P_{nm} \\
-(n+1) \sin{(m\phi)} P_{nm}
\end{array}
\right) & m < 0
\end{array}
\right.
\end{equation}

\begin{tabular}{>{$}c<{$} | >{$}c<{$} | >{$}c<{$}}
& \text{Vector residual } \boldsymbol{\epsilon}_i & \text{Scalar residual } f_i \\
\hline
\frac{\partial}{\partial g_{nm}} & -d\mathbf{B}_{nm}(\mathbf{r}_i) & \frac{1}{|| \mathbf{B}^{model}(\mathbf{r}_i; \mathbf{g},\mathbf{k})||} \mathbf{B}^{model}(\mathbf{r}_i; \mathbf{g},\mathbf{k}) \cdot d\mathbf{B}^{int}_{nm}(\mathbf{r}_i) \\
\hline
\frac{\partial}{\partial (\alpha,\beta,\gamma)} & R_q \left[ \frac{\partial}{\partial (\alpha,\beta,\gamma)} R_3(\alpha,\beta,\gamma) \right] \mathbf{B}^{VFM}_i & 0 \\
\hline
\frac{\partial}{\partial k(t)} & -d\mathbf{B}^{ext}(\mathbf{r}_i) & \frac{1}{|| \mathbf{B}^{model}(\mathbf{r}_i; \mathbf{g},\mathbf{k})||} \mathbf{B}^{model}(\mathbf{r}_i; \mathbf{g},\mathbf{k}) \cdot d\mathbf{B}^{ext}(\mathbf{r}_i)
\end{tabular}

\subsubsection{Optimization}

Since the cost function $\chi^2$ depends on both vector and scalar
residuals, we can write the Jacobian as
\begin{equation}
\mathbf{J} =
\left(
\begin{array}{ccccc}
\mathbf{J}_{MF}^{vec} & \mathbf{J}_{SV}^{vec} & \mathbf{J}_{SA}^{vec} & \mathbf{J}_{Euler}^{vec}(\mathbf{x}) & \mathbf{J}_{ext}^{vec}(\mathbf{x}) \\
\mathbf{J}_{MF}^{scal}(\mathbf{x}) & \mathbf{J}^{scal}_{SV}(\mathbf{x}) & \mathbf{J}^{scal}_{SA}(\mathbf{x}) & 0 & \mathbf{J}^{scal}_{ext}(\mathbf{x})
\end{array}
\right)
\end{equation}
where the top portion corresponds to vector residuals $\boldsymbol{\epsilon}_i$ and the bottom portion
corresponds to scalar residuals $f_i$. Even if the vector and scalar residuals are ``mixed'', so that
the Jacobian does not separate vertically as shown above, we can consider the above matrix
without loss of generality, since we can always rearrange the rows of the matrix as needed.
For simplicity, we define
\begin{equation}
\mathbf{J}_{int} =
\left(
\begin{array}{ccc}
\mathbf{J}_{MF} & \mathbf{J}_{SV} & \mathbf{J}_{SA}
\end{array}
\right)
\end{equation}
and note $\mathbf{J}_{SV} = t \mathbf{J}_{MF}$ and $\mathbf{J}_{SA} = \frac{1}{2} t^2 \mathbf{J}_{MF}$,
where $t$ is the timestamp of measurement $i$. The Jacobian then becomes
\begin{equation}
\mathbf{J} =
\left(
\begin{array}{ccc}
\mathbf{J}_{int}^{vec} & \mathbf{J}_{Euler}^{vec}(\mathbf{x}) & \mathbf{J}_{ext}^{vec}(\mathbf{x}) \\
\mathbf{J}_{int}^{scal}(\mathbf{x}) & 0 & \mathbf{J}^{scal}_{ext}(\mathbf{x})
\end{array}
\right)
\end{equation}
Note that for vector residuals, $\mathbf{J}_{int}$ does not depend on the model parameters
$\mathbf{x}$. Also, the scalar residuals do not depend on the Euler angles, resulting in the
block of zeros in the above matrix. Additionally, while the matrices $\mathbf{J}_{int}^{vec}$
and $\mathbf{J}_{int}^{scal}(\mathbf{x})$ are dense, the rest of the Jacobian corresponding
to the Euler angles and external field parameters has a lot of sparse structure.
During the nonlinear least squares iterations, we require the normal equations matrix
$\mathbf{J}^T \mathbf{J}$. This matrix can be computed very efficiently by accounting
for the sparse structure in the above matrix. Writing it all out, we have:
\begin{equation}
\mathbf{J}^T \mathbf{J} =
\left(
\begin{array}{ccc}
\mathbf{J}_{int}^T \mathbf{J}_{int}^{vec} + \mathbf{J}_{int}^T(\mathbf{x}) \mathbf{J}_{int}^{scal}(\mathbf{x}) & X & X \\
\mathbf{J}_{Euler}^T(\mathbf{x}) \mathbf{J}_{int} & \mathbf{J}_{Euler}^T(\mathbf{x}) \mathbf{J}_{Euler}(\mathbf{x}) & X \\
\mathbf{J}_{ext}^T(\mathbf{x}) \mathbf{J}_{int}^{vec} + \mathbf{J}_{ext}^T(\mathbf{x}) \mathbf{J}_{int}^{scal}(\mathbf{x})  & \mathbf{J}_{ext}^T(\mathbf{x}) \mathbf{J}_{Euler}(\mathbf{x}) & \mathbf{J}_{ext}^T(\mathbf{x}) \mathbf{J}_{ext}^{vec}(\mathbf{x}) + \mathbf{J}_{ext}^T(\mathbf{x}) \mathbf{J}_{ext}^{scal}(\mathbf{x}) \\
\end{array}
\right)
\end{equation}
The $X$ entries above indicate that the matrix is symmetric and so only the lower half needs to
be computed. The $(1,1)$ term $\mathbf{J}_{int}^T \mathbf{J}_{int}^{vec}$ can be precomputed
since it does not depend on $\mathbf{x}$, which saves significant computations during the
iteration.

\end{document}
