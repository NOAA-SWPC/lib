\documentclass{article}

\usepackage{amsmath}

\begin{document}

\section{Introduction}

\section{Internal field}
\label{sec:internal}

The internal scalar potential is given in geocentric spherical coordinates by
\begin{equation}
V_{int}(r,\theta,\phi) = a \sum_{nm} \left( \frac{a}{r} \right)^{n+1} \left( g_{nm} \cos{m \phi} + h_{nm} \sin{m \phi} \right)
P_{nm}(\cos{\theta})
\end{equation}
Defining $\mathbf{B} = - \nabla V$ gives
\begin{align}
B_x = -B_{\theta} = \frac{1}{r} \frac{\partial V}{\partial \theta} &= \sum_{nm} \left( \frac{a}{r} \right)^{n+2}
\left( g_{nm} \cos{m \phi} + h_{nm}\sin{m \phi} \right) \frac{\partial}{\partial \theta} P_{nm}(\cos{\theta}) \\
B_y = B_{\phi} = -\frac{1}{r \sin{\theta}} \frac{\partial V}{\partial \phi} &= \frac{1}{\sin{\theta}} \sum_{nm}
\left( \frac{a}{r} \right)^{n+2} m \left( g_{nm} \sin{m \phi} - h_{nm} \cos{m \phi} \right) P_{nm}(\cos{\theta}) \\
B_z = -B_r = \frac{\partial V}{\partial r} &= -\sum_{nm} (n + 1) \left( \frac{a}{r} \right)^{n+2}
\left( g_{nm} \cos{m \phi} + h_{nm}\sin{m \phi} \right) P_{nm}(\cos{\theta})
\end{align}

\section{Complex Internal field}
\label{sec:internal_complex}

The complex internal scalar potential is given in geocentric spherical coordinates by
\begin{equation}
V_{int}(r,\theta,\phi) = a \sum_{nm} \left( \frac{a}{r} \right)^{n+1} g_{nm} Y_{nm}(\theta,\phi)
\end{equation}
where the coefficients $g_{nm}$ are complex and $Y_{nm} = P_{nm}(\cos{\theta}) \exp{im\phi}$.
Defining $\mathbf{B} = - \nabla V$ gives
\begin{align}
B_x = -B_{\theta} = \frac{1}{r} \frac{\partial V}{\partial \theta} &= \sum_{nm} \left( \frac{a}{r} \right)^{n+2}
g_{nm} \frac{\partial}{\partial \theta} Y_{nm} \\
B_y = B_{\phi} = -\frac{1}{r \sin{\theta}} \frac{\partial V}{\partial \phi} &= -\frac{1}{\sin{\theta}} \sum_{nm}
im \left( \frac{a}{r} \right)^{n+2} g_{nm} Y_{nm} \\
B_z = -B_r = \frac{\partial V}{\partial r} &= -\sum_{nm} (n + 1) \left( \frac{a}{r} \right)^{n+2} g_{nm} Y_{nm}
\end{align}

\section{External field}

The external scalar potential is given in geocentric spherical coordinates by
\begin{equation}
V_{ext}(r,\theta,\phi) = a \sum_{nm} \left( \frac{r}{a} \right)^n \left( q_{nm} \cos{m \phi} + k_{nm} \sin{m \phi} \right)
P_{nm}(\cos{\theta})
\end{equation}

The magnetic field components are
\begin{align}
B_x &= \sum_{nm} \left( \frac{r}{a} \right)^{n-1} \left( q_{nm} \cos{m \phi} + k_{nm} \sin{m \phi} \right)
\frac{\partial}{\partial \theta} P_{nm}(\cos{\theta}) \\
B_y &= \frac{1}{\sin{\theta}} \sum_{nm} \left( \frac{r}{a} \right)^{n-1} m
\left( q_{nm} \sin{m \phi} - k_{nm} \cos{m \phi} \right) P_{nm}(\cos{\theta}) \\
B_z &= \sum_{nm} n \left( \frac{r}{a} \right)^{n-1} \left( q_{nm} \cos{m \phi} + k_{nm} \sin{m \phi} \right)
P_{nm}(\cos{\theta})
\end{align}

\section{Complex External field}

The complex external scalar potential is given in geocentric spherical coordinates by
\begin{equation}
V_{ext}(r,\theta,\phi) = a \sum_{nm} k_{nm} \left( \frac{r}{a} \right)^n Y_{nm}(\theta,\phi)
\end{equation}

The magnetic field components are
\begin{align}
B_x &= \sum_{nm} k_{nm} \left( \frac{r}{a} \right)^{n-1} \frac{\partial}{\partial \theta} Y_{nm} \\
B_y &= -\frac{1}{\sin{\theta}} \sum_{nm} im k_{nm} \left( \frac{r}{a} \right)^{n-1} Y_{nm} \\
B_z &= \sum_{nm} n k_{nm} \left( \frac{r}{a} \right)^{n-1} Y_{nm}
\end{align}

\section{Modeling}

The penalty function which is minimized is
\begin{equation}
\chi^2 = \sum_{i=1}^{N_{vec}} \boldsymbol{\epsilon}_i \cdot \boldsymbol{\epsilon}_i + \sum_{i=1}^{N_{scal}} f_i^2
\end{equation}
where the residuals are
\begin{align}
\boldsymbol{\epsilon}_i & = R_q R_3(\alpha,\beta,\gamma) \mathbf{B}^{VFM}_i - \mathbf{B}^{model}(\mathbf{g},\mathbf{k}) \\
f_i & = || \mathbf{B}^{model}(\mathbf{g},\mathbf{k}) || - F_i
\end{align}
Here, $\mathbf{B}^{VFM}_i$ is the vector measurement in the VFM instrument frame and $F_i$ is the scalar field
measurement. The matrix $R_3(\alpha,\beta,\gamma)$ rotates a vector from the VFM frame to the CRF frame defined
by the star camera using the Euler angles $\alpha,\beta,\gamma$. The matrix $R_q$ then rotates from CRF to NEC
(see Olsen et al, 2013). The vector model is given by
\begin{equation}
\mathbf{B}^{model}(\mathbf{g},\mathbf{r}) = \mathbf{B}^{int}(\mathbf{g}) + \mathbf{B}^{crust} + \mathbf{B}^{ext,pomme} + \mathbf{B}^{ext,correction}(\mathbf{k})
\end{equation}
where $\mathbf{B}^{int}(\mathbf{g})$ is an internal field model defined in Sec.~\ref{sec:internal},
$\mathbf{B}^{crust}$ is the MF7 crustal field model from degree 16 to 133,
$\mathbf{B}^{ext,pomme}$ is the POMME-8 external field model, and $\mathbf{B}^{ext,correction}(\mathbf{k})$ is a daily
ring current correction, parameterized as
\begin{equation}
\mathbf{B}^{ext,correction}(\mathbf{k}) = k(t) \left( 0.7 \mathbf{B}^{ext,dipole} + 0.3 \mathbf{B}^{int,dipole} \right)
\end{equation}
where $\mathbf{B}^{ext,dipole}$ and $\mathbf{B}^{int,dipole}$ are degree 1 external and internal dipole fields,
whose coefficients are aligned with the main field dipole (ie: $g_{10},g_{11},h_{11}$). For each day, there is
one coefficient $k(t)$ for that day which is determined through the least
squares minimization. The crustal field term, $\mathbf{B}^{crust}$, can
optionally be set to zero, in order to fit a high degree crustal field
in $\mathbf{B}^{int}$.

\end{document}
