\documentclass{article}

\usepackage{amsmath}

\newcommand{\B}{\mathbf{B}}
\newcommand{\J}{\mathbf{J}}
\newcommand{\R}{\mathbf{r}}
\newcommand{\Bsh}{\B_{pol}^{sh}}
\newcommand{\qsh}{q_{nm}^{sh}}
\newcommand{\qi}{q_{nm}^i}
\newcommand{\q}{q_{nm}}
\newcommand{\psish}{\psi_{nm}^{sh}}

\begin{document}

\section{Introduction}

The geomagnetic field can be decomposed into toroidal and poloidal
components:
\begin{equation}
\B = \B_{tor} + \B_{pol}
\label{eqn:Btot}
\end{equation}
where
\begin{align}
\B_{tor} &= \wedge \Phi \\
\B_{pol} &= \nabla \times \wedge \Psi
\end{align}
Similarly, the current density can everywhere be decomposed as:
\begin{equation}
\mu_0 \J = \mu_0 \J_{tor} + \mu_0 \J_{pol}
\end{equation}
where
\begin{align}
\mu_0 \J_{tor} &= \wedge Q \\
\mu_0 \J_{pol} &= \nabla \times \wedge \Phi
\end{align}
Here, we see immediately that $\nabla \times \B_{tor} = \mu_0 \J_{pol}$.
By imposing $\nabla \times \B_{pol} = \mu_0 \J_{tor}$, we can derive
\begin{equation}
\nabla^2 \Psi = -Q
\end{equation}

\subsection{Spherical harmonic expansions of the toroidal and poloidal scalars}

Now we see that the full magnetic field $\B$ and current density $\J$
can be specified by three scalar functions $\Phi,\Psi,Q$. We can
expand these scalars in a basis of spherical harmonics:
\begin{align}
\Phi(r,\theta,\phi) = \sum_{nm} \phi_{nm}(r) Y_{nm}(\theta,\phi) \\
\Psi(r,\theta,\phi) = R \sum_{nm} \psi_{nm}(r) Y_{nm}(\theta,\phi) \\
Q(r,\theta,\phi) = \frac{1}{R} \sum_{nm} q_{nm}(r) Y_{nm}(\theta,\phi) \label{eqn:Q}
\end{align}
Here, $R$ is a reference radius introduced so that
$\phi_{nm}(r),\psi_{nm}(r),q_{nm}(r)$ all have the same units (nT).
This makes numerical computations easier. The spherical harmonics
are defined as
\begin{equation}
Y_{nm}(\theta,\phi) = S_{n|m|}(\cos{\theta}) \exp{im\phi}
\end{equation}
where $S_{nm}(\cos{\theta})$ are the Schmidt semi-normalized associated
Legendre functions.

\subsection{Spherical harmonic expansion of the magnetic field}

With the spherical harmonic expansions of the toroidal and poloidal
scalars defined above, we can expand the toroidal and poloidal magnetic
field as follows.
\begin{align}
\B_{tor} &= \sum_{nm} \phi_{nm}(r)
\left[
\begin{array}{c}
0 \\
-\frac{im}{\sin{\theta}} Y_{nm} \\
\partial_{\theta} Y_{nm}
\end{array}
\right] \label{eqn:Btor} \\
\B_{pol} &= -\frac{R}{r} \sum_{nm}
\left[
\begin{array}{c}
n(n+1) \psi_{nm}(r) Y_{nm} \\
\partial_r \left( r \psi_{nm}(r) \right) \partial_{\theta} Y_{nm} \\
\frac{im}{\sin{\theta}} \partial_r \left( r \psi_{nm}(r) \right) Y_{nm}
\end{array}
\right] \label{eqn:Bpol}
\end{align}

\subsection{Spherical harmonic expansion of the current}

Similarly, the current density can be expanded as
\begin{align}
\mu_0 \J_{tor} &= \frac{1}{R} \sum_{nm} q_{nm}(r)
\left[
\begin{array}{c}
0 \\
-\frac{im}{\sin{\theta}} Y_{nm} \\
\partial_{\theta} Y_{nm}
\end{array}
\right] \\
\mu_0 \J_{pol} &= -\frac{1}{r} \sum_{nm}
\left[
\begin{array}{c}
n(n+1) \phi_{nm}(r) Y_{nm} \\
\partial_r \left( r \phi_{nm}(r) \right) \partial_{\theta} Y_{nm} \\
\frac{im}{\sin{\theta}} \partial_r \left( r \phi_{nm}(r) \right) Y_{nm}
\end{array}
\right]
\end{align}

\subsection{Relating $\Psi$ and $Q$}

From the relation $\nabla^2 \Psi = -Q$, one can show
\begin{equation}
\psi_{nm}(r) = \int_{0}^{\infty} G_n(r,s) \left( \frac{s}{R} \right)^2 q_{nm}(s) ds
\label{eqn:psinm1}
\end{equation}
where the Green's function $G_n(r,s)$ is given by
\begin{equation}
G_n(r,s) = \frac{1}{2n + 1}
\left\{
\begin{array}{cc}
\frac{r^n}{s^{n+1}}, & r < s \\
\frac{s^n}{r^{n+1}}, & r > s
\end{array}
\right.
\end{equation}
If the toroidal current is restricted to the shell $S(a,c)$, so that
$q_{nm}(r) = 0$ outside the shell, Eq.~\eqref{eqn:psinm1} becomes
\begin{equation}
\psi_{nm}(r) = \int_{a}^{c} G_n(r,s) \left( \frac{s}{R} \right)^2 q_{nm}(s) ds
\label{eqn:psinm2}
\end{equation}
Since the quantity $\partial_r (r \psi_{nm}(r))$ appears in
the expansion for $\B_{pol}$, it is instructive to examine this
expression for the case $r > c$. If we wish to know the poloidal
scalar above the current shell, the Green's function becomes
$G_n(r,s) = \frac{1}{2n+1} \frac{s^n}{r^{n+1}}$ and
\begin{equation}
r \psi_{nm}(r > c) = \frac{1}{r^n} \frac{1}{2n+1} \int_a^c s^n \left( \frac{s}{R} \right)^2 q_{nm}(s) ds
\label{eqn:psiabove}
\end{equation}
The integral does not depend on $r$, and so we find for the case $r > c$:
\begin{equation}
\partial_r (r \psi_{nm}(r)) = -n \psi_{nm}(r)
\label{eqn:psi_above}
\end{equation}
Similarly, if we are below the current shell, with $r < a$, we will find:
\begin{equation}
\partial_r (r \psi_{nm}(r)) = (n+1) \psi_{nm}(r)
\label{eqn:psi_below}
\end{equation}
Inside the current shell, with $a \le r \le c$, the integral will
depend on $r$ and so the expression for $\partial_r(r \psi_{nm}(r))$
becomes more complex. To summarize, the poloidal fields due to
internal and external toroidal current sources are given by
\begin{align}
\B_{pol}^i = \nabla \times \wedge \Psi^i &=
-\frac{R}{r} \sum_{nm} n \psi_{nm}^i(r)
\left[
\begin{array}{c}
(n+1) Y_{nm} \\
-\partial_{\theta} Y_{nm} \\
\frac{-im}{\sin{\theta}} Y_{nm}
\end{array}
\right] \label{eqn:Bpoli} \\
\B_{pol}^e = \nabla \times \wedge \Psi^e &=
-\frac{R}{r} \sum_{nm} (n+1) \psi_{nm}^e(r)
\left[
\begin{array}{c}
n Y_{nm} \\
\partial_{\theta} Y_{nm} \\
\frac{im}{\sin{\theta}} Y_{nm}
\end{array}
\right] \label{eqn:Bpole}
\end{align}

\section{Fitting poloidal and toroidal fields to satellite data}

Assume we have a satellite which samples the geomagnetic field
within a shell $S(r_1,r_2)$. The shell will have some thickness to it
due to the satellite's elliptical orbit, as well as orbital decay
over its lifetime. The magnetic field $\B$ seen by the satellite can
be expressed as
\begin{equation}
\B = \B_{pol}^i + \B_{pol}^e + \B_{pol}^{sh} + \B_{tor}
\end{equation}
where $\B_{pol}^i$ is the poloidal field due to toroidal currents flowing
internally to the satellite shell, $\B_{pol}^e$ is the poloidal field
due to toroidal currents flowing externally to the satellite shell,
$\B_{pol}^{sh}$ is the poloidal field due to toroidal currents flowing
in the satellite shell, and $\B_{tor}$ is the toroidal field due
to poloidal currents flowing in the satellite shell. $\B_{tor}$ does
not need to be separated into internal/external/within components,
since poloidal currents flowing outside the satellite shell cannot
be seen by the satellite (Backus).

\subsection{$\B_{pol}^i$ and $\B_{pol}^e$}

The fields $\B_{pol}^i$ and $\B_{pol}^e$ are potential fields in the
satellite shell, since their current densities are 0 in $S(r_1,r_2)$.
Therefore,
\begin{align}
\B_{pol}^i &= -\nabla V^i \\
\B_{pol}^e &= -\nabla V^e
\end{align}

The scalar potential fields are given by
\begin{align}
V^i(r,\theta,\phi) &= R \sum_{nm} g_{nm} \left( \frac{R}{r} \right)^{n+1} Y_{nm}(\theta,\phi) \\
V^e(r,\theta,\phi) &= R \sum_{nm} k_{nm} \left( \frac{r}{R} \right)^n Y_{nm}(\theta,\phi)
\end{align}
Since these potential fields are also poloidal, they satisfy
Eqs.~\eqref{eqn:Bpoli}-\eqref{eqn:Bpole}, and so for the internal case:
\begin{align}
-\nabla V^i &= \nabla \times \wedge \Psi^i \\
\sum_{nm} g_{nm} \left( \frac{R}{r} \right)^{n+2}
\left[
\begin{array}{c}
(n+1) Y_{nm} \\
-\partial_{\theta} Y_{nm} \\
\frac{-im}{\sin{\theta}} Y_{nm}
\end{array}
\right]
&=
-\frac{R}{r} \sum_{nm} n \psi_{nm}^i(r)
\left[
\begin{array}{c}
(n+1) Y_{nm} \\
-\partial_{\theta} Y_{nm} \\
\frac{-im}{\sin{\theta}} Y_{nm}
\end{array}
\right]
\end{align}
By inspection, we find
\begin{equation}
\psi_{nm}^i(r) = -\frac{1}{n} \left( \frac{R}{r} \right)^{n+1} g_{nm}
\label{eqn:psi_g}
\end{equation}
For the external polidal field, we have
\begin{align}
-\nabla V^e &= \nabla \times \wedge \Psi^e \\
-\sum_{nm} k_{nm} \left( \frac{r}{R} \right)^{n-1}
\left[
\begin{array}{c}
n Y_{nm} \\
\partial_{\theta} Y_{nm} \\
\frac{im}{\sin{\theta}} Y_{nm}
\end{array}
\right]
&=
-\frac{R}{r} \sum_{nm} (n+1) \psi_{nm}^e(r)
\left[
\begin{array}{c}
n Y_{nm} \\
\partial_{\theta} Y_{nm} \\
\frac{im}{\sin{\theta}} Y_{nm}
\end{array}
\right]
\end{align}
and so by inspection
\begin{equation}
\psi_{nm}^e(r) = \frac{1}{n+1} \left( \frac{r}{R} \right)^n k_{nm}
\end{equation}

\subsection{Visualizing current flow}

For the internal field $\B_{pol}^i$, we would additionally like to visualize the flow of the
toroidal current system (EEJ and Sq), and so we relate the toroidal current scalars $\q(r)$
and potential coefficients $g_{nm}$ by assuming a spherical sheet current flowing at some radius
$b$. Since the sheet current has no thickness, we parameterize the current scalars as
\begin{equation}
\q(r) = \q b \delta(r - b)
\label{eqn:qnm}
\end{equation}
where the factor of $b$ is inserted so that $\q$ has the same units as $\q(r)$ (nT).
Using this expression in Eq.~\eqref{eqn:psiabove} for the poloidal scalar for a current
flowing beneath the satellite shell yields
\begin{equation}
\psi_{nm}^i(r) = \frac{\q}{2n + 1} \left( \frac{b}{r} \right)^{n+1} \left( \frac{b}{R} \right)^2
\label{eqn:psi_q}
\end{equation}
and combining this with Eq.~\eqref{eqn:psi_g} yields
\begin{equation}
g_{nm} = -\frac{n}{2n+1} \q \left( \frac{b}{R} \right)^{n+3}
\label{eqn:g_q}
\end{equation}
If we choose the current shell radius $b$ high enough that we have satellite measurements
below the shell (ie: CHAMP measurements near 250-300 km altitude), we must also consider
the current shell as an external source. In this case, we find
\begin{equation}
\psi_{nm}^e(r) = \frac{\q}{2n + 1} \left( \frac{r}{b} \right)^n \left( \frac{b}{R} \right)^2
\label{eqn:psie_q}
\end{equation}
and also
\begin{equation}
k_{nm} = q_{nm} \frac{n+1}{2n + 1} \left( \frac{R}{b} \right)^{n-2}
\end{equation}

To solve for the $\q$ in the least-squares inversion, we would use the parameterizations
\begin{align}
\B_{pol}(r > b) &= -\frac{R}{r} \left( \frac{b}{R} \right)^2 \sum_{nm} \q \frac{n}{2n + 1}
\left( \frac{b}{r} \right)^{n+1}
\left[
\begin{array}{c}
(n+1) Y_{nm} \\
-\partial_{\theta} Y_{nm} \\
-\frac{im}{\sin{\theta}} Y_{nm}
\end{array}
\right] \\
\B_{pol}(r < b) &= -\frac{R}{r} \left( \frac{b}{R} \right)^2 \sum_{nm} \q \frac{n+1}{2n + 1}
\left( \frac{r}{b} \right)^n
\left[
\begin{array}{c}
n Y_{nm} \\
\partial_{\theta} Y_{nm} \\
\frac{im}{\sin{\theta}} Y_{nm}
\end{array}
\right]
\end{align}
depending whether the satellite measurement is above or below the current shell at $r = b$.

To parameterize $\B_{pol}^i$ in the least squares inversion, we could either use
$\B_{pol}^i = \nabla \times \wedge \Psi^i$ with Eq.~\eqref{eqn:psi_q}, and solve
for the $\q$ directly, or use $\B_{pol}^i = -\nabla V^i$, solve for the $g_{nm}$
and use Eq.~\eqref{eqn:g_q} to recover the $\q$. Once the $\q$ are determined,
the sheet current density (in A/m) flowing at the radius $b$ is given by
\begin{equation}
\mu_0 \mathbf{K}_{tor}^i = \frac{b}{R} \sum_{nm} \q
\left[
\begin{array}{c}
0 \\
-\frac{im}{\sin{\theta}} Y_{nm} \\
\partial_{\theta} Y_{nm}
\end{array}
\right]
\end{equation}

\subsubsection{The current stream function}

To visualize the equivalent current flow at $r = b$, it is most useful to draw contours
of the current stream function $\chi^i$. For a toroidal current, the current stream function is
defined by (see Haines, 1994, eq 25)
\begin{equation}
\J_{tor} = \nabla \times \left[ \hat{r} \chi^i(\theta,\phi) \right]
\end{equation}
so with our definitions,
\begin{equation}
\chi^i(\theta,\phi) = -\frac{Qb}{\mu_0} = -\frac{b}{\mu_0} \frac{b}{R} \sum_{nm} \q Y_{nm}(\theta,\phi)
\end{equation}
with the $\q$ defined in Eq.~\eqref{eqn:qnm}. Using $\mu_0 = 400 \pi$ nT / (kA km$^{-1}$),
we see $\chi^i$ will have units of kA. The contours of the $\chi^i$ stream function will illustrate
current flow, counterclockwise around a maximum, and clockwise around a minimum.

\subsection{$\B_{pol}^{sh}$}

The shell poloidal field is due to toroidal currents flowing inside the satellite shell $S(a,c)$. Here,
we will expand the toroidal current scalar radial dependence as a Taylor series:
\begin{equation}
q_{nm}(r) = \sum_{j=0}^J q_{nm}^{(j)} \left( \frac{r - r_0}{R} \right)^j
\end{equation}
where we have absorbed the $1/j!$ term into $q_{nm}^{(j)}$ and added the reference radius
to keep the radial term dimensionless. $r_0$ is the mean altitude of the current shell,
ie: $(a+c)/2$. The corresponding poloidal scalar can be expressed
as
\begin{align}
\psi_{nm}(r) &= \sum_j q_{nm}^{(j)} \left( A_n^{(j)}(r) + B_n^{(j)}(r) \right) \\
\partial_r \left( r \psi_{nm}(r) \right) &= \sum_j q_{nm}^{(j)} \left( -n A_n^{(j)}(r) + (n+1) B_n^{(j)}(r) \right)
\end{align}
where
\begin{align}
A_n^{(j)}(r) &= \frac{1}{2n + 1} \left( \frac{r}{R} \right)^{j+2} \int_{a/r}^1 s^{n+2} \left( s - \frac{r_0}{r} \right)^j ds \\
B_n^{(j)}(r) &= \frac{1}{2n + 1} \left( \frac{r}{R} \right)^{j+2} \int_{1}^{c/r} \frac{1}{s^{n-1}} \left( s - \frac{r_0}{r} \right)^j ds
\end{align}
Finally, the poloidal magnetic field is given by
\begin{equation}
\B_{pol}^{sh} = -\frac{R}{r} \sum_{nmj} q_{nm}^{(j)}
\left[
\begin{array}{c}
n(n+1) \left( A_n^{(j)}(r) + B_n^{(j)}(r) \right) Y_{nm} \\
\left( -n A_n^{(j)}(r) + (n+1) B_n^{(j)}(r) \right) \partial_{\theta} Y_{nm} \\
\frac{im}{\sin{\theta}} \left( -n A_n^{(j)}(r) + (n+1) B_n^{(j)}(r) \right) Y_{nm}
\end{array}
\right]
\end{equation}

The current density is given by
\begin{equation}
\mu_0 \J_{tor}^{sh} = \frac{1}{R} \sum_{nmj} q_{nm}^{(j)}
\left( \frac{r - r_0}{R} \right)^j
\left[
\begin{array}{c}
0 \\
-\frac{im}{\sin{\theta}} Y_{nm} \\
\partial_{\theta} Y_{nm}
\end{array}
\right]
\end{equation}

\section{Misc}

Possible functions for the toroidal current which go to zero (0th and 1st
derivatives) on the current shell $S(a,c)$:
\begin{align}
q_{nm}(r) &= q_{nm} \cos{\left[ \frac{\pi}{c-a} \left( r - \frac{1}{2}(c+a) \right) \right]}^2 \\
q_{nm}(r) &= q_{nm} (r-a)^2 (r-c)^2
\end{align}

\section{Real case}

When using real-valued spherical harmonic basis functions, Eqs.~\eqref{eqn:Btot}-\eqref{eqn:Q}
remain the same. But we define a new real-valued basis function
\begin{equation}
Y_{nm}(\theta,\phi) = \left\{
\begin{array}{cc}
S_{nm}(\cos{\theta}) \cos{(m \phi)} & m \ge 0 \\
S_{n|m|}(\cos{\theta}) \sin{(|m| \phi)} & m < 0
\end{array}
\right.
\end{equation}
for the spherical harmonic expansion. For these real-valued functions, we still
have the identity $r^2 \nabla^2 Y_{nm} = -n(n+1) Y_{nm}$, which is needed to
derive the magnetic field vectors. They are
\begin{align}
\B_{tor} &= \sum_{nm} \phi_{nm}(r)
\left[
\begin{array}{c}
0 \\
-\frac{1}{\sin{\theta}} \partial_{\phi} Y_{nm} \\
\partial_{\theta} Y_{nm}
\end{array}
\right] \\
\B_{pol} &= -\frac{R}{r} \sum_{nm}
\left[
\begin{array}{c}
n(n+1) \psi_{nm}(r) Y_{nm} \\
\partial_r \left( r \psi_{nm}(r) \right) \partial_{\theta} Y_{nm} \\
\frac{1}{\sin{\theta}} \partial_r \left( r \psi_{nm}(r) \right) \partial_{\phi} Y_{nm}
\end{array}
\right]
\end{align}
The current densities are
\begin{align}
\mu_0 \J_{tor} &= \frac{1}{R} \sum_{nm} q_{nm}(r)
\left[
\begin{array}{c}
0 \\
-\frac{1}{\sin{\theta}} \partial_{\phi} Y_{nm} \\
\partial_{\theta} Y_{nm}
\end{array}
\right] \\
\mu_0 \J_{pol} &= -\frac{1}{r} \sum_{nm}
\left[
\begin{array}{c}
n(n+1) \phi_{nm}(r) Y_{nm} \\
\partial_r \left( r \phi_{nm}(r) \right) \partial_{\theta} Y_{nm} \\
\frac{1}{\sin{\theta}} \partial_r \left( r \phi_{nm}(r) \right) \partial_{\phi} Y_{nm}
\end{array}
\right]
\end{align}

\section{Identities}

Here are some useful identities involving the toroidal and poloidal differential operators.

\begin{align}
\wedge &= \R \times \nabla \\
\wedge f &= -\nabla \times \left( \R f \right) \\
\nabla \times \wedge f &= \R \nabla^2 f - \nabla \left( f + \R \cdot \nabla f \right) \\
\wedge \cdot \left( \nabla \times \wedge \right) &= 0 \\
\R \cdot \wedge &= 0 \\
\nabla \cdot \wedge &= 0 \\
\nabla \cdot \left( \nabla \times \wedge \right) &= 0 \\
\wedge f(\theta,\phi) &=
\left[
\begin{array}{c}
0 \\
-\frac{1}{\sin{\theta}} \partial_{\phi} f \\
\partial_{\theta} f
\end{array}
\right]
\end{align}

\end{document}
