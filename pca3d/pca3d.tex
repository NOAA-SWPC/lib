\documentclass{article}

\usepackage{amsmath}

\begin{document}

\title{3D PCA Methods for Ionospheric and Induction Studies}
\author{Patrick Alken}

\maketitle

\section{Preliminaries}

Astrid has provided a 1-month run from TIEGCM, which contains the full
3D current vector $\mathbf{J}(\mathbf{r},t)$ on a spherical grid at hourly timesteps throughout the month.
The run is for February 2009, and with hourly grids, we have a total of 672
time steps.

The grid region coveres all latitudes and longitudes, with an altitude range from
about 110 km to 950 km. There are 18 vertical layers provided, with unequal spacing
(more dense spacing near the E-region to better capture the sharp changes in EEJ/Sq).

I will denote grid locations in spherical coordinates $(r,\theta,\phi)$ by triplets
$(i,j,k)$, where $i$ corresponds to the radial location, $j$ to the colatitude location,
and $k$ to the longitude location.

\section{Dividing the time span into shorter segments and finding the spectra of each segment}

In order to see how the different frequency components in the currents change over time
(i.e. due to seasonal and annual effects), we divide the full 1-month time period into shorter
``time segments''. This is done by defining two parameters, (1) the size of the segment and
(2) the shift forward. In the previous study I used a segment size of 2 days with a shift of 1 day.
Currently, I have changed this to a segment size of 5 days with a shift of 1 day. The reason for
moving to 5 days, is when looking at spectrograms of the currents, a 2 day window seems insufficient
to resolve all of the daily subharmonics for each window. 5 days seems to do a decent job, but perhaps
we need to discuss this further.

In either case, we can define $T$ to be the total number of time segments. Then for each segment
$l, 1 \le l \le T$, we Fourier transform the $l$-th segment. If the original time-domain signal
at the spatial grid location $(ijk)$ is denoted
$$
\mathbf{J}_{ijk}(t),
$$
then I will denote the Fourier transform of the set of samples belonging to time segment $l$ as
$$
\mathbf{\tilde{J}}_{ijk;l}(\omega)
$$

\section{Building the PCA matrix}

\noindent
Similar to our previous work, I will closely follow the methodology described in Egbert and Booker, 1989.

\vspace{5 mm}
\noindent
Now, for a given frequency, we want to find all the correlations (linearities) between the Fourier
amplitudes of each pair of time segments and also each pair of spatial grid locations. Then, using
PCA we can change basis to one in which each basis vector is statistically (and linearly) independent
of the others. The hope is that a small number of the most dominant basis vectors will be
sufficient to fully describe what the TIEGCM run is telling us.

\noindent
To do this, I define a complex column vector
\begin{equation}
\mathbf{X}_l(\omega_p) =
\left[
\begin{array}{c}
(\tilde{J}_r)_{ijk;l}(\omega_p) \\
(\tilde{J}_{\theta})_{ijk;l}(\omega_p) \\
(\tilde{J}_{\phi})_{ijk;l}(\omega_p)
\end{array}
\right]
\end{equation}
which has length $3 N$, where $N = N_r N_{\theta} N_{\phi}$ is the size of the
spatial grid. The idea here is to collect the Fourier amplitudes (corresponding
to frequency $\omega_p$) for the single time segment $l$ from all spatial grid
locations together into one column vector.

\noindent
Then we can define a complex $3N \times T$ matrix $\mathbf{X}$ as
\begin{equation}
\mathbf{X}(\omega_p) =
\left[
\begin{array}{cccc}
\mathbf{X}_1(\omega_p) & \mathbf{X}_2(\omega_p) & \dots & \mathbf{X}_T(\omega_p)
\end{array}
\right]
\end{equation}
Here, we have just collected each column vector for each time segment into one matrix.
The $3N \times 3N$ spectral density matrix is then
\begin{equation}
\mathbf{S}(\omega_p) = \frac{1}{T} \mathbf{X}(\omega_p) \mathbf{X}^{\dag}(\omega_p)
\end{equation}
Forming this outer product essentially multiplies and sums the products of each
pair of $\mathbf{X}_l(\omega_p)$, which tells us the covariance between the Fourier amplitudes
corresponding to the same frequency $\omega_p$ between each pair of time segments.

\noindent
Computing the eigenvectors of $\mathbf{S}(\omega_p)$, or equivalently the left singular
vectors of $\mathbf{X}(\omega_p)$, will give us the dominant modes of variability
for the desired frequency $\omega_p$.

\noindent
The way I have constructed this present study, by directly using the full spatial current
grids (rather than a compact spherical harmonic representation of them), the singular
vectors can be directly plotted on a lat/lon map to show us the spatial representation
of the modes. This is what is done in the accompanying slides.

\end{document}
